\documentclass{homework}

\name{Zhang Chi} % Replace (Student Name) with your name.
\id{2022010754}
\term{2024 Spring}
\course{Introduction to Theoretical Computer Science}
\hwnum{6}

%\hwname{(Name)}          % Uncomment and replace (Name) with the type of the
                          % homework (e.g, Assignment, Problem Set, etc.) if you
                          % don't want the document to be labeled as "Homework."
%\problemname{(Name)}     % Uncomment and replace (Name) with the desired label
                          % for problems created with the problem environment.
%\solutionname{(Name)}    % Uncomment and replace (Name) with the desired label
                          % for solutions created with the solution environment.

% Load any other packages you need here.

\begin{document}

\begin{problem}
  Prove that if $\P = \NP$, then $\NP = \coNP$.
\end{problem}

\begin{solution}
  Suppose $\P=\NP$. Then for any problem $A\in\NP$, $A\in \P$. So there exists a TM $T_1$ that can decide $A$ in polynomial time. We construct a new TM $T_2$, which simulates $T_1$ and gives the opposite answer (when $T_1$ accepts, $T_2$ rejects, and vice versa). By definition $T_2$ will decide $\bar{A}$. Since $T_1$ runs in polynomial time, $T_2$ also runs in polynomial time. Therefore, $\bar{A}\in \P=\NP$. So $A\in\coNP$, indicating $\NP\subset\coNP$. Meanwhile, for any $A\in\coNP$, we have $\bar{A}\in\NP\subset\coNP$. So $A\in\NP$, which means $\coNP\subset\NP$. Therefore, $\NP=\coNP$.
\end{solution}

\begin{problem}
  For every $\TWOSAT$ instance $\varphi$ of $n$ variables, define graph
  $G_{\varphi}$ of $2n$ vertices as follows.
  For each variable $x_{i}$ in $\varphi$, $G_{\varphi}$ has two vertices labeled
  by $x_{i}$ and $\neg x_{i}$ respectively.
  There is a directed edge $\ell_{i} \to \ell_{j}$ if
  $(\neg \ell_{i}) \lor \ell_{j}$ or $\ell_{j} \lor (\neg \ell_{i})$ is a clause
  of $\varphi$.
  For notational convenience, for literal $\ell_{i} = \neg x_{k_{i}}$,
  $\neg \ell_{i}$ is defined to be $x_{k_{i}}$.
  Prove that $\varphi$ is unsatisfiable if and only if there exist
  paths from $x_{j}$ to $\neg x_{j}$ and from $\neg x_{j}$ to $x_{j}$ in
  $G_{\varphi}$ for some $j$.
  Use the above fact to show that $\TWOSAT \in \P$.
\end{problem}

\begin{solution}
  $\varphi$ is satisfiable if and only if there exists an assignment of values to variables $x_i$ ensuring all clauses to be true. If there exists a path from  $x_j$ to $\neg x_j$, suppose the vertices on the path are $\ell_1,\ell_2,...,\ell_k$. Then it means that $\neg x_j\lor \ell_1,\neg\ell_1\lor\ell_2,...,\neg\ell_k\lor \neg x_j$ are all clauses in $\varphi$. Now we attempt to assign values to variables to satisfy all these clauses. If $x_j=1$, then since $\neg x_j\lor \ell_1=1$, $\ell_1= 1$. Similarly we can deduce that $\ell_2=1,...,\ell_k=1$. Then $\neg\ell_k=0$, which contradicts with $\neg\ell_k\lor \neg x_j=1$. So $\varphi \text{ satisfiable}\Rightarrow x_j=0$. Meanwhile, if there also exists a path from $\neg x_j$ to $x_j$, we can similarly deduce that $x_j=1$, which arises contradiction. So the existence of both paths implies $\varphi$ is unsatisfiable. 
  
  If $\forall j$, $G_\varphi$ does not contain circles including $x_j$ and $\neg x_j$, we will prove that $\varphi$ is satisfiable. $\forall j$, if there exists a path from $x_j$ to $\neg x_j$, we need to assign $x_j$ to $0$ to avoid contradiction. Similarly, if there exists a path from $\neg x_j$ to $x_j$, we need to assign $x_j$ to $1$. For all reachable paths $\ell_i \to\ell_j$ in $G_\varphi$, if $\ell_i=1$ assign $\ell_j=1$. After assigning all variables we can assign, randomly assign a variable to $0$ or $1$ and repeat the process. Such assigning process will guarantee all clauses are made true, and will satisfy $\varphi$. We will prove that this process will not raise contradiction. 
  
  In the first assigning stage, if some $\ell_1,\ell_2$ all need to be assigned $1$, they are generated from some vertices that reach there own negatives. Assume $\ell_1\to y$ and $\ell_2\to \neg y$, and $\neg\ell_1\to\ell_1,\neg\ell_2\to\ell_2$. Then there happens to be $y\to\neg\ell_2$ and $\neg y\to\neg\ell_1$, which makes a circle containing $y$ and $\neg y$. In the second assigning stage, if $\ell$ is assigned $1$ and $\ell\to y$ and $\ell\to \neg y$, then there are also $\neg y \to\neg\ell$ and $y\to\neg\ell$, which makes a path $\ell\to\neg\ell$. However, this indicates that $\ell$ should be assigned in the first assigning stage.

  Therefore, if there are no such circles, $\varphi$ is satisfiable. So we've proved the equivalence.

  Thus, we can consider a graph algorithm for $\TWOSAT$ using $G_\varphi$. For all variable $x$, test the reachability of $x\to\neg x$ and $\neg x \to x$. This test can be down by Dijkstra algorithm in polynomial time. If there exists some $x$ that the two path exist simultaneously, then $\varphi$ is not satisfiable. Otherwise, $\varphi$ is satisfiable. We need to test $n$ times in total, indicating the whole algorithm is still in polynomial time.

\end{solution}

\begin{problem}
  The Lehmer's theorem states that a natural number $n$ is a prime number if and
  only if the following two conditions hold:
  \begin{enumerate}
    \item There is number $a$ such that $a^{n-1} \equiv 1 \pmod{n}$.
    \item For every prime factor $q$ of $n-1$, $a^{(n-1)/q} \not\equiv 1 \pmod{n}$.
  \end{enumerate}
  Use this theorem to show that $\PRIME \in \NP \cap \coNP$.
  (Hint: To prove $\PRIME \in \NP$, you may need to use recursively defined
  witness.)
\end{problem}

\begin{solution}
  First we will prove $\PRIME\in\coNP$. For any integer $a$, we can verify it's not a prime by giving a divider of $a$, say $q$, where $q\neq 1$ and $q\neq a$, as proof string. In polynomial time we calculate $a\%q$, if it's zero it means that $a$ has a divider and thus is not a prime.

  Then we will prove $\PRIME \in\NP$. We will prove that there exists a polynomial $t(n)\ge n^3$, for any prime $p$, we can verify that $p$ is a prime in $O(t(\log p))$ time. Use induction on $p$. For $p=2$, it's obviously true. Assume it's true for all primes less Than $p$, we will prove it's true for $p$.

  We construct the verifier using the given property of prime numbers. The proof string contains the number $a$ and the prime factor decomposition of $p-1$: $p-1 = \prod_i q_i^{\alpha_i}$. The verifier will do the following:
  \begin{enumerate}
    \item Verify $a^{p-1} \equiv 1 \pmod{p}$. Using the fast-exponentiation algorithm this can be done in $O(\log p)$ time. 
    \item For the given prime factor decomposition of $p-1$, verify that it is indeed the correct decomposition. As $\prod_i q_i^{\alpha_i}=p-1$, we have $\sum_i \log(q_i)\le\log(p-1)$. \begin{enumerate}
      \item By the induction hypothesis, we can verify $q_i$ indeed are primes in $O(t(\log q_i))$ time. Verifying all $q_i$ can be done in $O(\sum_i t(\log q_i))\le O(t(\sum_i\log q_i))\le O(t(\log(p-1)))\le O(t(\log p))$, where the second inequality holds because $(\sum a_i)^m \le \sum a_i^m$ for positive $a_i$ and $m$.
      \item Verify the equality $\prod_i q_i^{\alpha_i}=p-1$ holds. This can be done in at most $\sum_i (\log(p))^{3}\le (\log(p))^{3}$, since there are at most $\log p$ prime factors.
    \end{enumerate}
    \item For all the given $q_i$, verify that $a^{(p-1)/q_i} \not\equiv 1 \pmod{p}$. Each can be done in $O(\log p)$ time, and since there are at most $\log p$ prime factors we can finish the verification in $O((\log p)^2)$ time.
  \end{enumerate}
  Therefore, the total verification will cost at most $O(t(\log p) + (\log p)^{3}) = O(t(\log p))$ time, as $t(n)\ge n^3$, which completes the inductive step's proof,

  By induction, we know that for all $p$ prime, we can verify it in polynomial time, indicating $\PRIME\in\NP$. So $\PRIME\in\NP\cap\coNP$.
  
\end{solution}

\end{document}
