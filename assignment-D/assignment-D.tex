\documentclass{homework}

\name{Zhang Chi} % Replace (Student Name) with your name.
\id{2022010754}
\term{2024 Spring}
\course{Introduction to Theoretical Computer Science}
\hwnum{13}

%\hwname{(Name)}          % Uncomment and replace (Name) with the type of the
                          % homework (e.g, Assignment, Problem Set, etc.) if you
                          % don't want the document to be labeled as "Homework."
%\problemname{(Name)}     % Uncomment and replace (Name) with the desired label
                          % for problems created with the problem environment.
%\solutionname{(Name)}    % Uncomment and replace (Name) with the desired label
                          % for solutions created with the solution environment.

% Load any other packages you need here.

\begin{document}

\begin{problem}
  Prove that for every $\AM$ protocol for a language $A$, if Merlin and Arthur
  repeat the protocol $k$ times in parallel (Arthur runs $k$ independent random
  strings for each message and accepts only if all $k$ copies accept), then the
  probability that Arthur accepts $x \not\in A$ is at most $1/2^{k}$.
  (Recall that an $\AM$ protocol starts with Arthur sending the random string
  and Merlin replying a witness.
  You should not assume that the Merlin message for parallelized protocol is
  independent for each copy in your proof.)
\end{problem}

\begin{solution}
The AM protocol satisfies that if $x\not\in A$ then $\Pr(\langle P,V\rangle(x)=1)\le 1/2$. Consider each time the Verifier generates random string $r_i$, and denote this simulation as $\langle P,V\rangle_i$. We need to prove that for all $x\not\in A$, $\Pr(\bigwedge_{i=1}^k \langle P,V\rangle_i(x)=1)\le 1/2^k$. Use induction. Assume that $\Pr(\bigwedge_{i=1}^{k-1} \langle P,V\rangle_i(x)=1)\le 1/2^{k-1}$, and prove the $k$ case below. If $\Pr(\bigwedge_{i=1}^{k-1} \langle P,V\rangle_i(x)=1)\le 1/2^k$, then obviously the conditional probability is not greater than 1, and the proof should be done. Thus, we only need to consider the case when $\Pr(\bigwedge_{i=1}^{k-1} \langle P,V\rangle_i(x)=1)\ge 1/2^k$. Below we will prove that $\Pr(\langle P,V\rangle_k(x)=1|\bigwedge_{i=1}^{k-1} \langle P,V\rangle_i(x)=1)\le 1/2$

If $\Pr(\langle P,V\rangle_k(x)=1|\bigwedge_{i=1}^{k-1} \langle P,V\rangle_i(x)=1) > 1/2$ for some $x\not\in A$: Because in Merlin-Arthur mode the Verifier has no private information, its every move can be simulated. Consider a $P'$ as follows: On any input random string $r_k$:
\begin{enumerate}
  \item Simulate $\langle P,V\rangle_i$ for $i=1,2,...,k-1$.
  \item If some simulations reject, rewind from step 1.
  \item Until all simulations accpet. we simulate $P$ with the $r_i$ s generated in the last simulation, and the new input $r_k$.
\end{enumerate}
Now we will prove thie $P'$ can always halt and is correct. If $x\in A$ then al $\langle P,V\rangle_i$ will accept in the first round and then $P'$ will accept. If $x\not\in A$, from the induction hypothesis and the further assumption made above, the probability of the $k-1$ $\langle P,V\rangle_i$ all accepting is greater than $1/2^k$, so by continuing simulating step 3 will always be reached. (The probability of $P'$ not halting is $0$.)

However, using this $P'$ on the protocol we have $\Pr(\langle P',V\rangle(x)=1) = \Pr(\langle P,V\rangle_k(x)=1|\bigwedge_{i=1}^{k-1} \langle P,V\rangle_i(x)=1)>1/2$ where $x\not\in A$, which is a contradiction to the property of the AM protocol. Thus we have $\Pr(\langle P,V\rangle_k(x)=1|\bigwedge_{i=1}^{k-1} \langle P,V\rangle_i(x)=1) \le 1/2$ when $\Pr(\bigwedge_{i=1}^{k-1} \langle P,V\rangle_i(x)=1)>1/2^k$. Using the induction hypothesis we have $\Pr(\bigwedge_{i=1}^{k-1} \langle P,V\rangle_i(x)=1)\le 1/2^k$. By induction, the proof is done.

\end{solution}

\begin{problem}
  \begin{parts}
    \part\label{2.a} Explain why the following simulator does not work in establishing
    the zero-knowledge property of the protocol for $\GI$ discussed in the class.
    \begin{algorithmic}[1]
      \STATE{Choose $a\in \{0,1\}$ uniformly at random.}
      \STATE{Sample a random permutation $\pi$ and compute $G = \pi(G_a)$.}
      \STATE{Randomly sample $b \in \{0,1\}$.}
      \STATE{If $b = a$, output the transcript.
        Otherwise, rewind and start from the beginning.}
    \end{algorithmic}
    \part\label{2.b} Prove the zero-knowledge property of the protocol for $\GI$ discussed
    in the class formally.
  \end{parts}
\end{problem}

\begin{solution}
\ref{2.a} To establish the ZK property, the simulator needs to have $S(x) = \text{view}_{V^*}\langle P,V^*\rangle(x)$ for all $V^*$ and all $x\in\GI$. This simulator choose $b$ uniformly at random, but we can construct some $V^*$ to choose $b$ at another distribution. For example, let $\Pr(b=1)=0.9$. In $S$, the probability of the final chosen $b$ is $\Pr(b=1) = 0.5$ since $a$ and $b$ are both uniformly chosen and the result of $1$ and $0$ should be symmetric. Thus, the two distribution are distinguishable by $b$.

\ref{2.b} We need to prove that distribution $S(x) = \text{view}_{V^*}\langle P,V^*\rangle(x)$ for all $V^*$ and all $x\in\GI$. For any given $x\in\GI$ and $V^*$, $\text{view}_{V^*}\langle P,V^*\rangle(x)$ gives uniformly sampled graph $G$ isomorphic to the two graphs, a bit $b$ and a permutation between $G$ and $G_b$. For $S$, it gives a uniformly sampled $G$ as well, and gives a $b$ by simulating $V^*$, and then a permutation. We only need to prove the distribution of $b$ is the same, because the permutation can be uniquely determined by $G$ and $b$.

Say in $V^*$, $\Pr(b_{V^*}=1)=p$. Then in $S$, Say the sampled $a,b$ in the $i$-th iteration is $a^i,b^i_{V*}$, then we have:
\begin{align*}
  \Pr(b_S=1) &= \sum_{i=1}^\infty \Pr(a^i = 1, a^1 = ... = a^{i-1} = 0, b^i_{V^*}=1)\\
  &=  \sum_{i=1}^\infty (\frac{1}{2})^i\Pr(b^i_{V^*}=1)\\
  &= \bigl(\sum_{i=1}^\infty (\frac{1}{2})^i\bigr)\Pr(b_{V^*}=1) = \Pr(b_{V^*}=1)
\end{align*}
Thus, the distribution of $b$ is the same, and therefore the two distributions are indistinguishable.
\end{solution}

\end{document}
